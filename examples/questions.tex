\section*{Kemampuan Dasar}
\emph{Terdiri dari $10$ soal. Setiap soal yang dijawab benar bernilai $2$ poin dan tidak ada pengurangan untuk soal yang dijawab salah atau tidak dijawab(kosong).}
\begin{questions}
\question Misalkan $u_1,u_2,u_3$ barisan aritmetika dengan suku-suku bilangan positif. Jika $\frac{u_1+u_2}{u_3}=\frac{11}{21}$, maka nilai $\frac{u_2+u_3}{u_1}$ adalah ....
\question Koefisien suku $x^7$ pada penjabaran 
\[(1+x)(2+x^2)(3+x^3)(4+x^4)(5+x^5)\]
adalah ....
\question Diketahui fungsi $f$ terdefinisi di semua bilangan real selain $0$ dan $1$ memenuhi persamaan 
\[(x-1)f(-x)+\frac{1-x}{4x}f\left(\frac{1}{x}\right)=\frac{100(x^2+4)}{x}\]
Nilai dari $f(2)+f(3)+f(4)+\cdots+f(400)$ adalah ....
\question Diketahui bilangan bulat positif $A$ dan $B$ bila dibagi $5$ berturut-turut bersisa $2$ dan $3$. Sisa pembagian $A(A+1)+5B$ oleh $25$ adalah ....
\question Bilangan asli $n$ dikatakan \emph{menarik} jika terdapat suku banyak(polinom) dengan koefisien bulat $P(x)$ sehingga $P(7)=2021$ dan $P(n)=2045$. Banyak bilangan \textbf{prima} \emph{menarik} adalah ... 
\question Pada gambar di bawah ini sebuah persegi panjang dibagi dua menjadi dua buah persegi yang panjang sisinya $6$ cm. Luas total dari daerah yang diarsir adalah ...$\text{cm}^2$. 
\question Pada suatu lingkaran dengan jari-jari $r$, terdapat segiempat talibusur $ABCD$ dengan $AB=8$ dan $CD=5$. Sisia $AB$ dan $DC$ diperpanjang dan berpotongan di luar lingkaran di titik $p$. Jika $\angle APD=60^\circ$ dan $BP=6$, maka nilai $r^2$ adalah .... 
\question Bilangan $1,2,\ldots, 999$ digit-digitnya disusun membentuk angka baru $m$ dengan menuliskan semua digit bilangan-bilangan tadi dari kiri ke kanan. Jadi, $m=123\ldots 91011\ldots999$. Hasil penjumlahan digit ke $2021,2022,2023$ dari $m$ adalah ....
\question Diketahui ada enam pasang suami istri. Dari keenam pasangan tersebut akan dipilih enam orang secara acak. Banyaknya cara untuk memilih enam orang tersebut sehingga paling banyak terdapat sepasang suami istri adalah ....
\question Diketahui segitiga $ABC$ dengan $AB>AC$. Garis bagi sudut $BAC$ memotong sisi $BC$ di titik $D$. Titik $E$ dan $F$ berturut-turut terletak pada sisi $AC$ dan $AB$ sehingga $DE$ sejajar $AB$ dan $DF$ sejajar $AC$. Lingkaran luas segitiga $BCE$ memotong sisi $AB$ di titik $K$. Jika luas segitiga $CDE$ adalah $75$ dan luas segitiga $DEF$ adalah $85$, maka luas segiempat $DEKF$ adalah ....
\section*{Kemampuan Lanjut} 
\emph{Terdiri dari $10$ soal. Setiap soal yang dijawab benar bernilai $+4$ poin, dijawab salah bernilai $-1$ poin, dan tidak dijawab bernilai $0$ poin.}
\question Jika $a>1$ suatu bilangan asli sehingga hasil penjumlahan semua bilangan real $x$ yang memenuhi persamaan $\lfloor x\rfloor^2-2ax+a=0$ adalah $51$ adalah .... 
\question Diketahui bilangan real $a,b$, dan $c$ memenuhi ketaksamaan $|ax^2+bx+c|\leq 1$ untuk setiap $x\in R$ dengan $0\leq x\leq 1$. Nilai maksimal yang mungkin dari $21a+20b+19c$ adalah ....
\question Diberikan $x,y$, dan $n$ bilangan-bilangan asli yang memenuhi 
\[x^2+(y+2)x+(n+1)y=n^2+252\]
Nilai $y$ terbesar yang mungkin adalah .....
\question Jika dua digit terakhir dari $a^{777}$ adalah $777$, maka dua digit terakhir dari $a$ adalah ....
\question Bilangan asli ganjil $b$ terbesar sehingga barisan bilangan asli $a_n=n^2+19n+b$ memenuhi $FPB(a_n,a_{n+1})=FPB(a_{n+2},a_{n+1})$ untuk setiap bilangan asli $n$ adalah ....
\question Banyaknya fungsi(pemetaan) dari $A=\{1,2,3,4,5,6\}$ ke $B=\{7,8,9,10\}$ dengan syarat $7$ dan $8$ mempunyai prapeta, yaitu ada $x$ dan $y$ di $A$ sehingga $f(x)=7$ dan $f(y)=8$ adalah ....
\end{questions} 